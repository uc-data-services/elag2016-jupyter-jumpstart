
% Default to the notebook output style

    


% Inherit from the specified cell style.




    
\documentclass{article}

    
    
    \usepackage{graphicx} % Used to insert images
    \usepackage{adjustbox} % Used to constrain images to a maximum size 
    \usepackage{color} % Allow colors to be defined
    \usepackage{enumerate} % Needed for markdown enumerations to work
    \usepackage{geometry} % Used to adjust the document margins
    \usepackage{amsmath} % Equations
    \usepackage{amssymb} % Equations
    \usepackage{eurosym} % defines \euro
    \usepackage[mathletters]{ucs} % Extended unicode (utf-8) support
    \usepackage[utf8x]{inputenc} % Allow utf-8 characters in the tex document
    \usepackage{fancyvrb} % verbatim replacement that allows latex
    \usepackage{grffile} % extends the file name processing of package graphics 
                         % to support a larger range 
    % The hyperref package gives us a pdf with properly built
    % internal navigation ('pdf bookmarks' for the table of contents,
    % internal cross-reference links, web links for URLs, etc.)
    \usepackage{hyperref}
    \usepackage{longtable} % longtable support required by pandoc >1.10
    \usepackage{booktabs}  % table support for pandoc > 1.12.2
    \usepackage{ulem} % ulem is needed to support strikethroughs (\sout)
    

    
    
    \definecolor{orange}{cmyk}{0,0.4,0.8,0.2}
    \definecolor{darkorange}{rgb}{.71,0.21,0.01}
    \definecolor{darkgreen}{rgb}{.12,.54,.11}
    \definecolor{myteal}{rgb}{.26, .44, .56}
    \definecolor{gray}{gray}{0.45}
    \definecolor{lightgray}{gray}{.95}
    \definecolor{mediumgray}{gray}{.8}
    \definecolor{inputbackground}{rgb}{.95, .95, .85}
    \definecolor{outputbackground}{rgb}{.95, .95, .95}
    \definecolor{traceback}{rgb}{1, .95, .95}
    % ansi colors
    \definecolor{red}{rgb}{.6,0,0}
    \definecolor{green}{rgb}{0,.65,0}
    \definecolor{brown}{rgb}{0.6,0.6,0}
    \definecolor{blue}{rgb}{0,.145,.698}
    \definecolor{purple}{rgb}{.698,.145,.698}
    \definecolor{cyan}{rgb}{0,.698,.698}
    \definecolor{lightgray}{gray}{0.5}
    
    % bright ansi colors
    \definecolor{darkgray}{gray}{0.25}
    \definecolor{lightred}{rgb}{1.0,0.39,0.28}
    \definecolor{lightgreen}{rgb}{0.48,0.99,0.0}
    \definecolor{lightblue}{rgb}{0.53,0.81,0.92}
    \definecolor{lightpurple}{rgb}{0.87,0.63,0.87}
    \definecolor{lightcyan}{rgb}{0.5,1.0,0.83}
    
    % commands and environments needed by pandoc snippets
    % extracted from the output of `pandoc -s`
    \providecommand{\tightlist}{%
      \setlength{\itemsep}{0pt}\setlength{\parskip}{0pt}}
    \DefineVerbatimEnvironment{Highlighting}{Verbatim}{commandchars=\\\{\}}
    % Add ',fontsize=\small' for more characters per line
    \newenvironment{Shaded}{}{}
    \newcommand{\KeywordTok}[1]{\textcolor[rgb]{0.00,0.44,0.13}{\textbf{{#1}}}}
    \newcommand{\DataTypeTok}[1]{\textcolor[rgb]{0.56,0.13,0.00}{{#1}}}
    \newcommand{\DecValTok}[1]{\textcolor[rgb]{0.25,0.63,0.44}{{#1}}}
    \newcommand{\BaseNTok}[1]{\textcolor[rgb]{0.25,0.63,0.44}{{#1}}}
    \newcommand{\FloatTok}[1]{\textcolor[rgb]{0.25,0.63,0.44}{{#1}}}
    \newcommand{\CharTok}[1]{\textcolor[rgb]{0.25,0.44,0.63}{{#1}}}
    \newcommand{\StringTok}[1]{\textcolor[rgb]{0.25,0.44,0.63}{{#1}}}
    \newcommand{\CommentTok}[1]{\textcolor[rgb]{0.38,0.63,0.69}{\textit{{#1}}}}
    \newcommand{\OtherTok}[1]{\textcolor[rgb]{0.00,0.44,0.13}{{#1}}}
    \newcommand{\AlertTok}[1]{\textcolor[rgb]{1.00,0.00,0.00}{\textbf{{#1}}}}
    \newcommand{\FunctionTok}[1]{\textcolor[rgb]{0.02,0.16,0.49}{{#1}}}
    \newcommand{\RegionMarkerTok}[1]{{#1}}
    \newcommand{\ErrorTok}[1]{\textcolor[rgb]{1.00,0.00,0.00}{\textbf{{#1}}}}
    \newcommand{\NormalTok}[1]{{#1}}
    
    % Additional commands for more recent versions of Pandoc
    \newcommand{\ConstantTok}[1]{\textcolor[rgb]{0.53,0.00,0.00}{{#1}}}
    \newcommand{\SpecialCharTok}[1]{\textcolor[rgb]{0.25,0.44,0.63}{{#1}}}
    \newcommand{\VerbatimStringTok}[1]{\textcolor[rgb]{0.25,0.44,0.63}{{#1}}}
    \newcommand{\SpecialStringTok}[1]{\textcolor[rgb]{0.73,0.40,0.53}{{#1}}}
    \newcommand{\ImportTok}[1]{{#1}}
    \newcommand{\DocumentationTok}[1]{\textcolor[rgb]{0.73,0.13,0.13}{\textit{{#1}}}}
    \newcommand{\AnnotationTok}[1]{\textcolor[rgb]{0.38,0.63,0.69}{\textbf{\textit{{#1}}}}}
    \newcommand{\CommentVarTok}[1]{\textcolor[rgb]{0.38,0.63,0.69}{\textbf{\textit{{#1}}}}}
    \newcommand{\VariableTok}[1]{\textcolor[rgb]{0.10,0.09,0.49}{{#1}}}
    \newcommand{\ControlFlowTok}[1]{\textcolor[rgb]{0.00,0.44,0.13}{\textbf{{#1}}}}
    \newcommand{\OperatorTok}[1]{\textcolor[rgb]{0.40,0.40,0.40}{{#1}}}
    \newcommand{\BuiltInTok}[1]{{#1}}
    \newcommand{\ExtensionTok}[1]{{#1}}
    \newcommand{\PreprocessorTok}[1]{\textcolor[rgb]{0.74,0.48,0.00}{{#1}}}
    \newcommand{\AttributeTok}[1]{\textcolor[rgb]{0.49,0.56,0.16}{{#1}}}
    \newcommand{\InformationTok}[1]{\textcolor[rgb]{0.38,0.63,0.69}{\textbf{\textit{{#1}}}}}
    \newcommand{\WarningTok}[1]{\textcolor[rgb]{0.38,0.63,0.69}{\textbf{\textit{{#1}}}}}
    
    
    % Define a nice break command that doesn't care if a line doesn't already
    % exist.
    \def\br{\hspace*{\fill} \\* }
    % Math Jax compatability definitions
    \def\gt{>}
    \def\lt{<}
    % Document parameters
    \title{Markdown 101}
    
    
    

    % Pygments definitions
    
\makeatletter
\def\PY@reset{\let\PY@it=\relax \let\PY@bf=\relax%
    \let\PY@ul=\relax \let\PY@tc=\relax%
    \let\PY@bc=\relax \let\PY@ff=\relax}
\def\PY@tok#1{\csname PY@tok@#1\endcsname}
\def\PY@toks#1+{\ifx\relax#1\empty\else%
    \PY@tok{#1}\expandafter\PY@toks\fi}
\def\PY@do#1{\PY@bc{\PY@tc{\PY@ul{%
    \PY@it{\PY@bf{\PY@ff{#1}}}}}}}
\def\PY#1#2{\PY@reset\PY@toks#1+\relax+\PY@do{#2}}

\expandafter\def\csname PY@tok@c1\endcsname{\let\PY@it=\textit\def\PY@tc##1{\textcolor[rgb]{0.25,0.50,0.50}{##1}}}
\expandafter\def\csname PY@tok@si\endcsname{\let\PY@bf=\textbf\def\PY@tc##1{\textcolor[rgb]{0.73,0.40,0.53}{##1}}}
\expandafter\def\csname PY@tok@se\endcsname{\let\PY@bf=\textbf\def\PY@tc##1{\textcolor[rgb]{0.73,0.40,0.13}{##1}}}
\expandafter\def\csname PY@tok@gd\endcsname{\def\PY@tc##1{\textcolor[rgb]{0.63,0.00,0.00}{##1}}}
\expandafter\def\csname PY@tok@o\endcsname{\def\PY@tc##1{\textcolor[rgb]{0.40,0.40,0.40}{##1}}}
\expandafter\def\csname PY@tok@w\endcsname{\def\PY@tc##1{\textcolor[rgb]{0.73,0.73,0.73}{##1}}}
\expandafter\def\csname PY@tok@cm\endcsname{\let\PY@it=\textit\def\PY@tc##1{\textcolor[rgb]{0.25,0.50,0.50}{##1}}}
\expandafter\def\csname PY@tok@nf\endcsname{\def\PY@tc##1{\textcolor[rgb]{0.00,0.00,1.00}{##1}}}
\expandafter\def\csname PY@tok@nn\endcsname{\let\PY@bf=\textbf\def\PY@tc##1{\textcolor[rgb]{0.00,0.00,1.00}{##1}}}
\expandafter\def\csname PY@tok@mi\endcsname{\def\PY@tc##1{\textcolor[rgb]{0.40,0.40,0.40}{##1}}}
\expandafter\def\csname PY@tok@bp\endcsname{\def\PY@tc##1{\textcolor[rgb]{0.00,0.50,0.00}{##1}}}
\expandafter\def\csname PY@tok@gr\endcsname{\def\PY@tc##1{\textcolor[rgb]{1.00,0.00,0.00}{##1}}}
\expandafter\def\csname PY@tok@ow\endcsname{\let\PY@bf=\textbf\def\PY@tc##1{\textcolor[rgb]{0.67,0.13,1.00}{##1}}}
\expandafter\def\csname PY@tok@kp\endcsname{\def\PY@tc##1{\textcolor[rgb]{0.00,0.50,0.00}{##1}}}
\expandafter\def\csname PY@tok@kt\endcsname{\def\PY@tc##1{\textcolor[rgb]{0.69,0.00,0.25}{##1}}}
\expandafter\def\csname PY@tok@vg\endcsname{\def\PY@tc##1{\textcolor[rgb]{0.10,0.09,0.49}{##1}}}
\expandafter\def\csname PY@tok@mb\endcsname{\def\PY@tc##1{\textcolor[rgb]{0.40,0.40,0.40}{##1}}}
\expandafter\def\csname PY@tok@s\endcsname{\def\PY@tc##1{\textcolor[rgb]{0.73,0.13,0.13}{##1}}}
\expandafter\def\csname PY@tok@mh\endcsname{\def\PY@tc##1{\textcolor[rgb]{0.40,0.40,0.40}{##1}}}
\expandafter\def\csname PY@tok@mf\endcsname{\def\PY@tc##1{\textcolor[rgb]{0.40,0.40,0.40}{##1}}}
\expandafter\def\csname PY@tok@nt\endcsname{\let\PY@bf=\textbf\def\PY@tc##1{\textcolor[rgb]{0.00,0.50,0.00}{##1}}}
\expandafter\def\csname PY@tok@vi\endcsname{\def\PY@tc##1{\textcolor[rgb]{0.10,0.09,0.49}{##1}}}
\expandafter\def\csname PY@tok@gi\endcsname{\def\PY@tc##1{\textcolor[rgb]{0.00,0.63,0.00}{##1}}}
\expandafter\def\csname PY@tok@kc\endcsname{\let\PY@bf=\textbf\def\PY@tc##1{\textcolor[rgb]{0.00,0.50,0.00}{##1}}}
\expandafter\def\csname PY@tok@nl\endcsname{\def\PY@tc##1{\textcolor[rgb]{0.63,0.63,0.00}{##1}}}
\expandafter\def\csname PY@tok@k\endcsname{\let\PY@bf=\textbf\def\PY@tc##1{\textcolor[rgb]{0.00,0.50,0.00}{##1}}}
\expandafter\def\csname PY@tok@gp\endcsname{\let\PY@bf=\textbf\def\PY@tc##1{\textcolor[rgb]{0.00,0.00,0.50}{##1}}}
\expandafter\def\csname PY@tok@c\endcsname{\let\PY@it=\textit\def\PY@tc##1{\textcolor[rgb]{0.25,0.50,0.50}{##1}}}
\expandafter\def\csname PY@tok@sx\endcsname{\def\PY@tc##1{\textcolor[rgb]{0.00,0.50,0.00}{##1}}}
\expandafter\def\csname PY@tok@gt\endcsname{\def\PY@tc##1{\textcolor[rgb]{0.00,0.27,0.87}{##1}}}
\expandafter\def\csname PY@tok@go\endcsname{\def\PY@tc##1{\textcolor[rgb]{0.53,0.53,0.53}{##1}}}
\expandafter\def\csname PY@tok@cp\endcsname{\def\PY@tc##1{\textcolor[rgb]{0.74,0.48,0.00}{##1}}}
\expandafter\def\csname PY@tok@gs\endcsname{\let\PY@bf=\textbf}
\expandafter\def\csname PY@tok@nb\endcsname{\def\PY@tc##1{\textcolor[rgb]{0.00,0.50,0.00}{##1}}}
\expandafter\def\csname PY@tok@gu\endcsname{\let\PY@bf=\textbf\def\PY@tc##1{\textcolor[rgb]{0.50,0.00,0.50}{##1}}}
\expandafter\def\csname PY@tok@ne\endcsname{\let\PY@bf=\textbf\def\PY@tc##1{\textcolor[rgb]{0.82,0.25,0.23}{##1}}}
\expandafter\def\csname PY@tok@gh\endcsname{\let\PY@bf=\textbf\def\PY@tc##1{\textcolor[rgb]{0.00,0.00,0.50}{##1}}}
\expandafter\def\csname PY@tok@no\endcsname{\def\PY@tc##1{\textcolor[rgb]{0.53,0.00,0.00}{##1}}}
\expandafter\def\csname PY@tok@nd\endcsname{\def\PY@tc##1{\textcolor[rgb]{0.67,0.13,1.00}{##1}}}
\expandafter\def\csname PY@tok@sd\endcsname{\let\PY@it=\textit\def\PY@tc##1{\textcolor[rgb]{0.73,0.13,0.13}{##1}}}
\expandafter\def\csname PY@tok@s2\endcsname{\def\PY@tc##1{\textcolor[rgb]{0.73,0.13,0.13}{##1}}}
\expandafter\def\csname PY@tok@nv\endcsname{\def\PY@tc##1{\textcolor[rgb]{0.10,0.09,0.49}{##1}}}
\expandafter\def\csname PY@tok@m\endcsname{\def\PY@tc##1{\textcolor[rgb]{0.40,0.40,0.40}{##1}}}
\expandafter\def\csname PY@tok@sr\endcsname{\def\PY@tc##1{\textcolor[rgb]{0.73,0.40,0.53}{##1}}}
\expandafter\def\csname PY@tok@sb\endcsname{\def\PY@tc##1{\textcolor[rgb]{0.73,0.13,0.13}{##1}}}
\expandafter\def\csname PY@tok@sc\endcsname{\def\PY@tc##1{\textcolor[rgb]{0.73,0.13,0.13}{##1}}}
\expandafter\def\csname PY@tok@ss\endcsname{\def\PY@tc##1{\textcolor[rgb]{0.10,0.09,0.49}{##1}}}
\expandafter\def\csname PY@tok@cs\endcsname{\let\PY@it=\textit\def\PY@tc##1{\textcolor[rgb]{0.25,0.50,0.50}{##1}}}
\expandafter\def\csname PY@tok@mo\endcsname{\def\PY@tc##1{\textcolor[rgb]{0.40,0.40,0.40}{##1}}}
\expandafter\def\csname PY@tok@ni\endcsname{\let\PY@bf=\textbf\def\PY@tc##1{\textcolor[rgb]{0.60,0.60,0.60}{##1}}}
\expandafter\def\csname PY@tok@kd\endcsname{\let\PY@bf=\textbf\def\PY@tc##1{\textcolor[rgb]{0.00,0.50,0.00}{##1}}}
\expandafter\def\csname PY@tok@kr\endcsname{\let\PY@bf=\textbf\def\PY@tc##1{\textcolor[rgb]{0.00,0.50,0.00}{##1}}}
\expandafter\def\csname PY@tok@err\endcsname{\def\PY@bc##1{\setlength{\fboxsep}{0pt}\fcolorbox[rgb]{1.00,0.00,0.00}{1,1,1}{\strut ##1}}}
\expandafter\def\csname PY@tok@ge\endcsname{\let\PY@it=\textit}
\expandafter\def\csname PY@tok@na\endcsname{\def\PY@tc##1{\textcolor[rgb]{0.49,0.56,0.16}{##1}}}
\expandafter\def\csname PY@tok@nc\endcsname{\let\PY@bf=\textbf\def\PY@tc##1{\textcolor[rgb]{0.00,0.00,1.00}{##1}}}
\expandafter\def\csname PY@tok@il\endcsname{\def\PY@tc##1{\textcolor[rgb]{0.40,0.40,0.40}{##1}}}
\expandafter\def\csname PY@tok@vc\endcsname{\def\PY@tc##1{\textcolor[rgb]{0.10,0.09,0.49}{##1}}}
\expandafter\def\csname PY@tok@sh\endcsname{\def\PY@tc##1{\textcolor[rgb]{0.73,0.13,0.13}{##1}}}
\expandafter\def\csname PY@tok@kn\endcsname{\let\PY@bf=\textbf\def\PY@tc##1{\textcolor[rgb]{0.00,0.50,0.00}{##1}}}
\expandafter\def\csname PY@tok@s1\endcsname{\def\PY@tc##1{\textcolor[rgb]{0.73,0.13,0.13}{##1}}}

\def\PYZbs{\char`\\}
\def\PYZus{\char`\_}
\def\PYZob{\char`\{}
\def\PYZcb{\char`\}}
\def\PYZca{\char`\^}
\def\PYZam{\char`\&}
\def\PYZlt{\char`\<}
\def\PYZgt{\char`\>}
\def\PYZsh{\char`\#}
\def\PYZpc{\char`\%}
\def\PYZdl{\char`\$}
\def\PYZhy{\char`\-}
\def\PYZsq{\char`\'}
\def\PYZdq{\char`\"}
\def\PYZti{\char`\~}
% for compatibility with earlier versions
\def\PYZat{@}
\def\PYZlb{[}
\def\PYZrb{]}
\makeatother


    % Exact colors from NB
    \definecolor{incolor}{rgb}{0.0, 0.0, 0.5}
    \definecolor{outcolor}{rgb}{0.545, 0.0, 0.0}



    
    % Prevent overflowing lines due to hard-to-break entities
    \sloppy 
    % Setup hyperref package
    \hypersetup{
      breaklinks=true,  % so long urls are correctly broken across lines
      colorlinks=true,
      urlcolor=blue,
      linkcolor=darkorange,
      citecolor=darkgreen,
      }
    % Slightly bigger margins than the latex defaults
    
    \geometry{verbose,tmargin=1in,bmargin=1in,lmargin=1in,rmargin=1in}
    
    

    \begin{document}
    
    
    \maketitle
    
    

    
    \section{Markdown 101: Lesson Goals}\label{markdown-101-lesson-goals}

    \textbf{NOTE} - This lesson is adapted from Programming Historian's
\href{https://github.com/programminghistorian/jekyll/blob/gh-pages/lessons/getting-started-with-markdown.md}{Getting
Started with Markdown} * Intro to markdown, plain text-based syntax for
formatting docs * markdown is integrated into the jupyter notebook

    \subsection{What is markdown?}\label{what-is-markdown}

    \begin{itemize}
\tightlist
\item
  developed in 2004 by john gruber
\item
  a way of formatting text file
\item
  a perl utility for converting markdown into html
\end{itemize}

\textbf{plain text files} have many advantages of other formats 1. they
are readable on virt. all devices 2. withstood the test of time (legacy
word processing formats)

by using markdown you'll be able to produce files that are legible in
plain text and ready to be styled in other platforms

ex: * blogging engines, static site generators, sites like (github)
support markdown \& will render markdown into html * tools like pandoc
convert files into and out of markdown

    markdown files are saved in extention \texttt{.md} and can be opened in
text editors like textedit, notepad, sublime text, or vim

    We will be using the Jupyter notebook to write markdown in this lesson:

    create a new jupyter document
#### Headings
    \paragraph{Headings}\label{headings}

    Four levels of heading are avaiable in Markdown, and are indicated by
the number of \texttt{\#} preceding the heading text. Paste the
following examples into a code box.

    \begin{verbatim}
# First level heading
## Second level heading
### Third level heading
#### Fourth level heading
\end{verbatim}

    \section{First level heading}\label{first-level-heading}

\subsection{Second level heading}\label{second-level-heading}

\subsubsection{Third level heading}\label{third-level-heading}

\paragraph{Fourth level heading}\label{fourth-level-heading}

    First and second level headings may also be entered as follows:

\begin{verbatim}
First level heading
=======

Second level heading
----------
\end{verbatim}

    \section{First level heading}\label{first-level-heading}

\subsection{Second level heading}\label{second-level-heading}

    Notice how the Markdown syntax remains understandable even in the plain
text version.

    \paragraph{Paragraphs \& Line Breaks}\label{paragraphs-line-breaks}

Try typing the following sentence into the textbox:

\begin{verbatim}
Welcome to the Jupyter Jumpstart.

Today we'll be learning about Markdown syntax.
This sentence is separated by a single line break from the preceding one.
\end{verbatim}

    \textbf{This renders as:}

Welcome to Jupyter Jumpstart.

Today we'll be learning about Markdown syntax. This sentence is
separated by a single line break from the preceding one.

    \begin{itemize}
\tightlist
\item
  Paragraphs must be separated by an empty line
\item
  leave an empty line between \texttt{syntax} and \texttt{This}
\item
  some implementations of Markdown, single line breaks must also be
  indicated with two empty spaces at the end of each line
\end{itemize}

    \paragraph{Adding Emphasis}\label{adding-emphasis}

    \begin{itemize}
\tightlist
\item
  Text can be italicized by wrapping the word in \texttt{*} or
  \texttt{\_} symbols
\item
  bold text is written by wrapping the word in \texttt{**} or
  \texttt{\_}
\end{itemize}

    Try adding emphasis to a sentence using these methods:

\begin{verbatim}
I am **very** excited about the _Jupyter Jumpstart_ workshop.
\end{verbatim}

    \textbf{This renders as:}

I am \textbf{very} excited about the \emph{Jupyter Jumpstart} lessons.

    \paragraph{Making Lists}\label{making-lists}

    Markdown includes support for ordered and unordered lists. Try typing
the following list into the textbox:

\begin{verbatim}
Shopping List
----------
* Fruits
  * Apples
  * Oranges
  * Grapes
* Dairy
  * Milk
  * Cheese
\end{verbatim}

Indenting the \texttt{*} will allow you to created nested items.

    \textbf{This renders as:}

\subsection{Shopping List}\label{shopping-list}

\begin{itemize}
\tightlist
\item
  Fruits
\item
  Apples
\item
  Oranges
\item
  Grapes
\item
  Dairy
\item
  Milk
\item
  Cheese
\end{itemize}

    \textbf{Ordered lists} are written by numbering each line. Once again,
the goal of Markdown is to produce documents that are both legible as
plain text and able to be transformed into other formats.

\begin{verbatim}
To-do list
----------
1. Finish Markdown tutorial
2. Go to grocery store
3. Prepare lunch
\end{verbatim}

    \textbf{This renders as:}

\subsection{To-do list}\label{to-do-list}

\begin{enumerate}
\def\labelenumi{\arabic{enumi}.}
\tightlist
\item
  Finish Markdown tutorial
\item
  Go to grocery store
\item
  Prepare lunch
\end{enumerate}

    \textbf{This renders as:}

\subsection{To-do list}\label{to-do-list}

\begin{enumerate}
\def\labelenumi{\arabic{enumi}.}
\tightlist
\item
  Finish Markdown tutorial
\item
  Go to grocery store
\item
  Prepare lunch
\end{enumerate}

NOTE the proper order doesn't matter.

    \paragraph{Code Snippets}\label{code-snippets}

    \begin{itemize}
\tightlist
\item
  Represent code by wrapping snippets in back-tick characters like `````
\item
  for example \texttt{`\textless{}br\ /\textgreater{}`}
\item
  whole blocks of code are written by typing three backtick characters
  before and after each block
\end{itemize}

Try typing the following text into the textbox:

\begin{verbatim}
```html
<html>
    <head>
        <title>Website Title</title>
    </head>
    <body>
    </body>
</html>
```
\end{verbatim}

    \textbf{This renders as}:

\begin{verbatim}
    <html>
        <head>
            <title>Website Title</title>
        </head>
        <body>
        </body>
    </html>
\end{verbatim}

    \textbf{specific languages}

in jupyter you can specify specific lanauages for code syntax hylighting

example:

\begin{Shaded}
\begin{Highlighting}[]

\ControlFlowTok{for} \NormalTok{item }\OperatorTok{in} \NormalTok{collection:}
    \BuiltInTok{print}\NormalTok{(item)}
\end{Highlighting}
\end{Shaded}

note how the keywords in python are highlighted

    \paragraph{Blockquotes}\label{blockquotes}

Adding a \texttt{\textgreater{}} before any paragraph will render it as
a blockquote element.

Try typing the following text into the textbox:

\begin{verbatim}
> Hello, I am a paragraph of text enclosed in a blockquote. Notice how I am offset from the left margin. 
\end{verbatim}

\textbf{This renders as:}

\begin{quote}
Hello, I am a paragraph of text enclosed in a blockquote. Notice how I
am offset from the left margin.
\end{quote}

    \paragraph{Links}\label{links}

\begin{itemize}
\tightlist
\item
  Inline links are written by enclosing the link text in square brackets
  first, then including the URL and optional alt-text in round brackets
\end{itemize}

\texttt{For\ more\ tutorials,\ please\ visit\ the\ {[}Programming\ Historian{]}(http://programminghistorian.org/\ "Programming\ Historian\ main\ page").}

\textbf{This renders as:}

For more tutorials, please visit the
\href{http://programminghistorian.org/}{Programming Historian}.

    \paragraph{Images}\label{images}

Images can be referenced using \texttt{!}, followed by some alt-text in
square brackets, followed by the image URL and an optional title. These
will not be displayed in your plain text document, but would be embedded
into a rendered HTML page.

\texttt{!{[}Wikipedia\ logo{]}(http://upload.wikimedia.org/wikipedia/en/8/80/Wikipedia-logo-v2.svg\ "Wikipedia\ logo")}

\textbf{This renders as:}

\begin{figure}[htbp]
\centering
\includegraphics{http://upload.wikimedia.org/wikipedia/en/8/80/Wikipedia-logo-v2.svg}
\caption{Wikipedia logo}
\end{figure}

    \paragraph{Horizontal Rules}\label{horizontal-rules}

Horizontal rules are produced when three or more \texttt{-}, \texttt{*}
or \texttt{\_} are included on a line by themselves, regardless of the
number of spaces between them. All of the following combinations will
render horizontal rules:

\begin{verbatim}
___
* * *
- - - - - -
\end{verbatim}

\textbf{This renders as:}

\begin{longtable}[]{@{}l@{}}
\toprule
***\tabularnewline
\bottomrule
\end{longtable}

    \paragraph{Tables}\label{tables}

The core Markdown spec does not include tables; however, some sites and
applications use variants of Markdown that may include tables and other
special features.
\href{https://help.github.com/articles/github-flavored-markdown/}{GitHub
Flavored Markdown} is one of these variants, and is used to render
\texttt{.md} files in the browser on the GitHub site.

To create a table within GitHub, use pipes \texttt{\textbar{}} to
separate columns and hyphens \texttt{-} between your headings and the
rest of the table content. While pipes are only strictly necessary
between columns, you may use them on either side of your table for a
more polished look. Cells can contain any length of content, and it is
not necessary for pipes to be vertically aligned with each other.

\begin{verbatim}
| Heading 1 | Heading 2 | Heading 3 |
| --------- | --------- | --------- |
| Row 1, column 1 | Row 1, column 2 | Row 1, column 3|
| Row 2, column 1 | Row 2, column 2 | Row 2, column 3|
| Row 3, column 1 | Row 3, column 2 | Row 3, column 3|
\end{verbatim}

\textbf{This renders as:}

\begin{longtable}[]{@{}lll@{}}
\toprule
Heading 1 & Heading 2 & Heading 3\tabularnewline
\midrule
\endhead
Row 1, column 1 & Row 1, column 2 & Row 1, column 3\tabularnewline
Row 2, column 1 & Row 2, column 2 & Row 2, column 3\tabularnewline
Row 3, column 1 & Row 3, column 2 & Row 3, column 3\tabularnewline
\bottomrule
\end{longtable}

To specify the alignment of each column, colons \texttt{:} can be added
to the header row as follows:

\begin{verbatim}
| Left-aligned | Centered | Right-aligned |
| :-------- | :-------: | --------: |
| Apples | Red | 5000 |
| Bananas | Yellow | 75 |
\end{verbatim}

\textbf{This renders as:}

\begin{longtable}[]{@{}lcr@{}}
\toprule
Left-aligned & Centered & Right-aligned\tabularnewline
\midrule
\endhead
Apples & Red & 5000\tabularnewline
Bananas & Yellow & 75\tabularnewline
\bottomrule
\end{longtable}


    % Add a bibliography block to the postdoc
    
    
    
    \end{document}
